\documentclass[12pt,openright,oneside,a4paper,brazil]{abntex2}
\usepackage[utf8]{inputenc}
\usepackage[T1]{fontenc}
\usepackage{listing}
\usepackage{amsfonts}

\title{Projeto Interativo III - AirDrums}
\date{}
\author{\textbf{Caroline Bomfim Do Espirito Santo} (caroline.bomfim@hotmail.com.br), \\ \textbf{Mahaira Soares de Souza} (mahaira\_souza@hotmail.com), \\ \textbf{Thiago de Sousa Messias} (messiasthi@gmail.com). \\ \\ Ciência da Computação - Centro Universitário Senac}

\setlength{\parskip}{0.2cm} 
\setlrmarginsandblock{3cm}{2cm}{*}
\setulmarginsandblock{3cm}{2cm}{*}
\checkandfixthelayout

\begin{document}

\maketitle
 
\section*{Resumo}

\textit{De acordo com o proposto na disciplina Projeto Integrador III, a partir do estudo e aplicação de algoritmos relacionados a visão computacional, indiretamente atraves da biblioteca multiplataforma OPENCV, foi desenvolvido o AirDrums, uma bateria em jogo, onde o jogador deve seguir uma sequencia de passos para tocar a música e passar de fase, para isso, foi utizada a biblioteca gráfica allegro 5 e a linguagem de programação C-99.}\\

\textbf{Palavras-chave:} {OpenCV, allegro, biblioteca gráfica.

\section*{Abstract}

\textit{According to the proposed discipline "Projeto Integrador III" , from the study and application of algorithms related to computer vision , indirectly through the multiplatform library OpenCV was developed AirDrums , a battery in game, where the player must follow a sequence of steps to play the music and move from stage to this, was utizada the graphics library Allegro 5 and the programming language C-99.} \\

\textbf{Keywords:}\textit{OpenCV, allegro, graphics library.} 

\section*{Introdução}

Segundo pequisas, jogos que necessitam de movimentos rápidos e de raciocinio lógico por meio da interatividade das câmeras ajudam desenvolver a coordenação motora, agilidade, criatividade, percepção corporal e o equilíbrio físico e mental.

\section*{Revisão de Literatura}

No mercado atual existe uma diversidade de jogos envolvendo música, porém cada um segue uma linha de raciocionio e tecnologias diferentes. Pensando em jogos famosos, temos o GuitarHero, onde o jogador utilizando uma guitarra precisa acertar uma sequencia de notas para crescer no ranking e realizar a música, o RockBand, que segue a mesma linha de pensamento do GuitarHero, porém a bateria e o microfone são inclusos no pacote e o GarageBand, onde há uma reprodução da bateria em si. O AirDrums é uma mistura de todos os jogos citados, pois o objetivo é seguir uma sequencia de notas, onde a dificuldade aumenta gradativamente, o instrumento de uso são baquetas e o espaço, contudo o diferencial é a visão computacional, o jogador não precisa de muito para jogar, somente de baquetas (lápis, canetas e etc).

\section*{Desenvolvimento}
\textbf{HSV}:
O sistema de cores HSV formadas pelas componentes hue (matriz), saturation (saturação) e value (valor). O HSV também é conhecido como HSB (hue, saturation e brightness — matiz, saturação e brilho, respectivamente). 

\section*{Resultados}

\section*{Conclusão}

\begin{thebibliography}{6}

\bibitem *http://www.di.ubi.pt/~agomes/cg/teoricas/06-iluminacao.pdf \\
\newblock Capítulo Iluminação - Engenharia Informática

\bibitem *http://sidigicor.blogspot.com.br/2011/02/modelo-hsv.html \\
\newblok Explicação do Sistema de Cores HSV

\bibitem *http://www.ufrgs.br/engcart/PDASR/formcor.html \\
\newblock Ampla explicação do Sistema de Cores HSV

\end{thebibliography}

\end{document}